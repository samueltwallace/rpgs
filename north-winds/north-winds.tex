\documentclass[a4paper]{article}

\title{North Winds}

\author{Sam Wallace}

\date{Last updated \today}
\usepackage{accanthis}

\usepackage{graphicx}

\usepackage[margin=2cm]{geometry}

\usepackage{multicol}

\setlength{\columnsep}{1cm}

\usepackage{microtype}

\begin{document}

\maketitle
\newpage
\tableofcontents

\newpage

\begin{multicols}{2}
  
\section{Introduction}

This is a setting guide and toolkit for a medieval Scandinavian--inspired regions.
It is not a guide to particular places and peoples in such a setting, but instead gives ample information and hopefully inspiration so that a fitting setting may be developed quickly and easily. \\

This is not a history textbook on medieval Scandinavian life, but instead a collection of salient aspects of the examined culture that can inspire a familiar setting while holding on to some realistic accuracy.
\\

It includes an incomplete gameplay ruleset, but may be converted to the ruleset of a group's choice.
It should be generally compatible with modular and simpler rulesets; more modification may be necessary for rulesets of more detail. \\

This guide asks questions of a setting-builder rather than putting forth answers.
There are suggestions and options available, though details are not filled in and left to the builder's discretion.
The hope is that every setting inspired by the information provided here is unique and different. \\

If you have feedback on this, please contact me through appropriate channels and let me know what you think.
I appreciate all feedback.


\section{Basic Geographies of the Setting}

In the northern reaches of the world, there are four predominant geographical regions one may find: mountain ranges, taiga forest, polar seas, and the polar caps.

\subsection{Moutain Ranges}

Moutain ranges host a variety of life adjusted to the terrain and elevation. Small animals live amongst the many cracks and crevices, attracting bigger preadators. Verdant fields may span valleys between peaks, and caves in the mountains may stretch much deeper than at first glance. \\

Mountains are naturally a defensive geographical feature due to the difficult terrain. Transport through mountainous regions is slow and is often restricted to passes and paths. As a result, defensive human structures can be more common. \\

Consider the ecological and societal makeup in your montane regions.

\begin{itemize}
\item What larger monsters or beasts have come to pick off even the largest of mundane predators? How are they ajusted for a higher-altitude lifestyle?
\item What intelligent factions have taken up isolated residence among the peaks and valleys? Why have they decided to isolate themselves? How do they live in such an isolated environment?
\item What old structures remain in the mountains? Where do they lie among them?
\end{itemize}

Consider why a party of adventurous characters would visit such a place.

\begin{itemize}
\item Are there old ruins that might hide treasure or secrets?
\item Are there particular NPCs that have value to the party?
\item Are there rare biological ingredients here needed by the party?
\end{itemize}



\subsection{Taiga Forest}

Northern forests see much snow, and as a result are full of moisture in spring. This can cause a variety of environmental conditions, from deep forests to wetlands and lakes. Natural resources are generally plentiful here, and human settlements are common. \\

Consider a specific region in your taiga region.

\begin{itemize}
\item What is the ecological makeup of this region? What are its main geographical features?
\item What kinds of peoples live here? What technology do they have, what is their relationship with their environment, and what relationships do they have with other nearby peoples?
\item What natural resources are unique to this region? What types of society flourish in an area with as mild of a climate?
\end{itemize}

\subsection{Polar Seas} Interspersed equally by icebergs and islands, the northern seas see more activity than one would expect, though not all is friendly. \\

Raiders will board the few trade ships around and ransack local villages. There are many fishers at work in the region, both near the shore and farther out, but weather, beast, and cold keeps plentiful bounties on every cast. \\

What activities are present in your northern seas?

\begin{itemize}
\item Who do raiders and marauders attack? What other societies are present?
\item 
\end{itemize}

\subsection{Polar Caps}

There are no known standing structures beyond the northern seas. Only the coldest of winds blow. Snowy landscapes stretch interminably. Only the most foul of beasts live here.

\begin{itemize}
\item How have creatures adapted to the polar day and night cycles?
\item 
\end{itemize}

\section{Cultures and Societies}

Two main salient features of a northern society is that of isolation and cold.
These two 


\section{History and Religion}

General ideas of world history and its eschatology:

\begin{itemize}
\item Society began long ago when the world was still warm. People lived easily then.
\item The world became cold. Most think some great shift has happened in the magical energies of the world, indicating its slow decline into a frozen-over world.
\item The world will eventually end when the sun recedes into the sky and the world freezes over.
\end{itemize}

General religious forms:

\begin{itemize}
\item There is very little unification of religion. People pray to a spirit that will help them and that will listen.
\item Gods and spirits are very diverse and are often attributed specific domains. Many contradictory things can be said about the supposedly same being.
\item Many gods are seen as aloof and uncaring, and glimpses into their workings are highly prized. As a result, there are many divination practices and practitioners.
\end{itemize}

\section{Magic and Beasts}

Magic and wizards are rare. Magic is seen as meddling with that which is beyond humanity's domain, and will be punished. Most have never met a true practitioner of sorcery, and would fear to do so. \\

Inscribed runes are known to hold power. The symbol of the rune determines its power, but the symbols are unknown to all except those most versed in arcane lore. \\

Many beasts, some even believed to be magical, roam dark corners of the world. Though a cave bear may stay away from a human settlement, a troll will pick off any unlucky nighttime wanderers without a trace. There are even beasts that have learned or can imitate language, and these may lure hunters deeper and deeper into the forest. \\

\hrule

\appendix
\section{Rules for Play}

Have players come up with their characters in entirety, including background, any assets, connections to other players or NPCs, and a first objective. \\

Most risky situations can be resolved by evaluating a player's presentation of a character, thinking through their relevant skills and the situation at hand. If there is significant doubt, allow the player to roll two six-sided dice. \\

On a summed roll of 9 or greater, the action suceeds and the referee narrates the outcome. Otherwise the action fails, and the referee tells how. If it is decided the player has an advantage that does not guarantee a positive outcome, the player must roll only a 7 or greater. \\

Damage and health are tracked in deterministic hits. A standard person has 4 hits, which should be adjusted for armor and physique. A most basic weapon can deal a single hit, while a deadly weapon may do up to 4 hits when properly wielded. \\

The referee may and should use supplemental material for their advantage. Writing random encounter tables, a calendar of events outside the player's circumstances, and a hex-keyed map may all prove useful.

\section{Magical Powers}

If any among the players have the chance to learn magical secrets, the referee may use an option from below as a presentation of magic.

\begin{description}
\item[Runic Magic] A special rune, when written on an object or surface, may confer magical power. 
\item[Shamanic Rituals] Various rare natural objects, when collected and prepared in the proper way, can cause magical effects.
\item[Fetish Magic] Magical effects can also be contained within objects of particular forms, corresponding to the desired magic.
\item[Natural Magics] Many natural things are innately magical, but the more rare the more magical.
\end{description}

\subsection{Examples of Magic}

\begin{itemize}
\item A rune representing fire inscribed on a twig. When broken, a small flame alights that does not go out nor spread.
\item A meteorological ritual: to bring on deep fog for weeks. One must gather a collection of rare herbs and boil them in a stone pot until empty. The steam must be fanned into the air continuously in the desired direction of fog. 
\item Troll's blood will regenerate any lost flesh when prepared as a salve.
\item 
\end{itemize}

\section{Known Beasts}

NPCs that end up fighting can be outfitted with hits dealt and taken quite easily.
A small animal can likey take no more than one hit, while a heavily armored warrior or a monster can take 10 or 15. Give combatants interesting abilities, tactics, or powers. \\

Equip players with advantages, as they are weak, though never bend the numbers.
Allow for actions beyond a single attack, or impose effects other than taking hits. \\

\begin{itemize}
\item \textbf{Bear:} Deals 3 hits, can take 8 hits. Males are aggressive and females will retreat if severely wounded.
\item \textbf{Wolf: } Deals 2 hits, can take 4 hits. Travel in packs and pick off stragglers.
\item \textbf{Troll: } Deals 3 hits, can take 7 hits. Will retreat to heal quickly, and can mimic words.
\item \textbf{Darksnatch: } Deals 2 hits, can take 4 hits. Carries rope and drops out of trees to garrote victims before dragging them away. 
\end{itemize}



\end{multicols}
\end{document}
