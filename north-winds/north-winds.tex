\documentclass{article}

\title{North Winds \\
  \large Or Arctic Wolves}

\author{Sam Wallace}

\date{Last updated \today}
\usepackage{accanthis}

\usepackage{graphicx}

\usepackage[margin=0.5in]{geometry}

\usepackage{multicol}

\begin{document}

\maketitle
\newpage
\tableofcontents

\newpage

\begin{multicols}{2}
  
\section{Introduction}

This is a setting guide and toolkit for a Nordic--inspired hexcrawl.
It can be used either as-is, or modified to a referee's preferences.
It includes an incomplete ruleset, but may be converted to the ruleset of a group's choice.
It should be generally compatible with modular and simpler rulesets; more modification may be necessary for rulesets of more detail. \\

A number of generally known facts about the world are included, but many details are left out.
Referees may use this lack of details to add avenues into new adventures, or for incorporating desired cultural, historical, or societal elements at the referee's choice.
However, no critical information is excluded. \\

If you have feedback on this, please contact me through appropriate channels and let me know what you think.
I appreciate all feedback.


\section{Basic Ideas of the Setting}

Most of the world is cold. Its polar caps are covered in ice and only the very edges of the polar caps are ventured into, and only by the most brave of explorers.

\subsection{Biomes and Regions}

There are four latitudinal zones of the world:

\begin{description}
\item[Mountain Ranges] The region surrounding the equator. The slopes see snow even in summer.
\item[Taiga Forest] The region between equator and pole. Coniferous forests that will see snow, some even for most of the year.
\item[Polar Seas] Water surrounds the frozen poles and provides dangerous routes for sea travel and transportation.
\item[Polar Caps] Only the hardiest of man or beast can survive here.
\end{description}

There are, of course, local variations within these zones, but the world can be roughly chunked into these four regions for simplicity.
%% 
%% \begin{figure*}[h]
%%   \centering
%%   \includegraphics[width=\textwidth]{world-map}
%% \end{figure*}
%% 
\subsection{Moutain Ranges}

The majority of settled peoples live here. Villages and town survive among the valleys and foothills of the various ranges, and various passes connect their trade and transport. People live off the land, and the relative isolation of various settlements has prevented much centralization of power. \\

There are exceptions to this. Several kingdoms and castles have been erected or restored, and minor nobles hold their courts here.

\subsection{Taiga Forest}

Brave settlements stake out the line between the forest and the mountain ranges. These people are more interconnected, but under constant threat from the inhabitants of the forests. Resources here are more plentiful, but more risky to obtain.

\subsection{Polar Seas} Interspersed equally by icebergs and islands, the northern seas see more activity than one would expect, though not all is friendly. \\

Raiders will board the few trade ships around and ransack local villages. There are many fishers at work in the region, both near the shore and farther out, but weather, beast, and cold keeps plentiful bounties on every cast. \\

\subsection{Polar Caps}

There are no known standing structures beyond the northern seas. Only the coldest of winds blow. Snowy landscapes stretch interminably. Only the most foul of beasts live here.

\section{Cultures and Societies}

While some centralized power lies in the keeps and strongholds of more civilized regions, civilization is scarce and isolated beyond. \\

There are only a few distinct cultures. They are not unified and members within these culture often conflict, but by and large share similar features.

\begin{description}
\item[Kingdoms and their Towns] Organized towns and their hierarchical leadership. The lords, jarls, and kings of the various castles keep their subjects poor and themselves rich. 
\item[Frontier Settlements] Acting as independent villages, these settlements along the taiga and even a few hidden in the mountain ranges. They often have town militia and trade sparsely with other villages. Relationships with the nobility is strained.
\item[Tribes of the North] Several tribes of developing societies exist along the coast north of the Taiga. 
\end{description}


\section{History and Religion}

General ideas of world history and its eschatology:

\begin{itemize}
\item Society began long ago when the world was still warm. People lived easily then.
\item The world became cold. Most think some great shift has happened in the magical energies of the world, indicating its slow decline into a frozen-over world.
\item The world will eventually end when the sun recedes into the sky and the world freezes over.
\end{itemize}

General religious forms:

\begin{itemize}
\item There is very little unification of religion. People pray to a spirit that will help them and that will listen.
\item Gods and spirits are very diverse and are often attributed specific domains. Many contradictory things can be said about the supposedly same being.
\item Many gods are seen as aloof and uncaring, and glimpses into their workings are highly prized. As a result, there are many divination practices and practitioners.
\end{itemize}

\section{Magic and Beasts}

Magic and wizards are rare. Magic is seen as meddling with that which is beyond humanity's domain, and will be punished. Most have never met a true practitioner of sorcery, and would fear to do so. \\

Inscribed runes are known to hold power. The symbol of the rune determines its power, but the symbols are unknown to all except those most versed in arcane lore. \\

Many beasts, some even believed to be magical, roam dark corners of the world. Though a cave bear may stay away from a human settlement, a troll will pick off any unlucky nighttime wanderers without a trace. There are even beasts that have learned or can imitate language, and these may lure hunters deeper and deeper into the forest. \\

\appendix
\section{Rules for Play}

Have players come up with their characters in entirety, including background, any assets, connections to other players or NPCs, and a first objective. \\

Most risky situations can be resolved by evaluating a player's presentation of a character, thinking through their relevant skills and the situation at hand. If there is significant doubt, allow the player to roll two six-sided dice. \\

On a summed roll of 9 or greater, the action suceeds and the referee narrates the outcome. Otherwise the action fails, and the referee tells how. If it is decided the player has an advantage that does not guarantee a positive outcome, the player must roll only a 7 or greater. \\

Damage and health are tracked in deterministic hits. A standard person has 4 hits, which should be adjusted for armor and physique. A most basic weapon can deal a single hit, while a deadly weapon may do up to 4 hits when properly wielded.

The referee may and should use supplemental material for their advantage. Writing random encounter tables, a calendar of events outside the player's circumstances, and a hex-keyed map may all prove useful.

\section{Magical Powers}

If any among the players have the chance to learn magical secrets, the referee may use an option from below as a presentation of magic.

\begin{description}
\item[Runic Magic] A special rune, when written on an object or surface, may confer magical power. 
\item[Shamanic Rituals] Various rare natural objects, when collected and prepared in the proper way, can cause magical effects.
\item[Fetish Magic] Magical effects can also be contained within objects of particular forms, corresponding to the desired magic.
\item[Natural Magics] Many natural things are innately magical, but the more rare the more magical.
\end{description}

\subsection{Examples of Magic}

\begin{itemize}
\item A rune representing fire inscribed on a twig. When broken, a small flame alights that does not go out nor spread.
\item A meteorological ritual: to bring on deep fog for weeks. One must gather a collection of rare herbs and boil them in a stone pot until empty. The steam must be fanned into the air continuously in the desired direction of fog. 
\item Troll's blood will regenerate any lost flesh when prepared as a salve.
\item 
\end{itemize}

\section{Known Beasts}

NPCs that end up fighting can be outfitted with hits dealt and taken quite easily.
A small animal can likey take no more than one hit, while a heavily armored warrior or a monster can take 10 or 15. Give combatants interesting abilities, tactics, or powers. \\

Equip players with advantages, as they are weak, though never bend the numbers.
Allow for actions beyond a single attack, or impose effects other than taking hits. \\

\begin{itemize}
\item \textbf{Bear:} Deals 3 hits, can take 8 hits. Males are aggressive and females will retreat if severely wounded.
\item \textbf{Wolf: } Deals 2 hits, can take 4 hits. Travel in packs and pick off stragglers.
\item \textbf{Troll: } Deals 3 hits, can take 7 hits. Will retreat to heal quickly, and can mimic words.
\item \textbf{Darksnatch: } Deals 2 hits, can take 4 hits. Carries rope and can drop out of trees to garrote victims before dragging them away. 
\end{itemize}



\end{multicols}
\end{document}
