\documentclass{article}

\usepackage[margin=0.5in]{geometry}

\pagestyle{empty}

\setlength{\parindent}{0pt}

\newcommand{\indented}[1]{\hfill\begin{minipage}[t]{0.7\textwidth} #1 \end{minipage}}

\usepackage[usefilenames,RMstyle=Light,SSstyle=Light,TTstyle=Light,DefaultFeatures={Ligatures=Common}]{plex-otf} %
\renewcommand*\familydefault{\ttdefault} %% Only if the base font of the document is to be monospaced


\begin{document}
\noindent APRS(1) \hfill User Manuals \hfill APRS(1) \\

NAME
\begin{quote}aprs - Astronaut Psychology Research and Survival\end{quote}

USAGE
\begin{quote}
aprs [-h]
\end{quote}

DESCRIPTION
\begin{quote}
  aprs is a program to research and sustain healthy astronaut psychology when in isolation in deep space. It asks a series of questions to the astronaut, which the astronaut responds to. Answers are recorded, and follow-up questions are posed by the computer. Detailed usage instructions can be found in the following sections.
\end{quote}

OPTIONS
\begin{quote}
  -h\qquad Display this message.
\end{quote}

FILES
\begin{quote}
  $\sim$/.cache/aprs.log
  \begin{quote}
    Log of aprs calculations and reponses, persistent across sessions.
  \end{quote}
\end{quote}

BUGS
\begin{quote}
  Due to computational resources on computer embedded in flight ships and the complexity of aprs, the computer only has the capability of asking 4 questions per year. Answering more than 4 questions per year risks system overload and total collapse of computational resources, essentially leaving the astronaut abandoned deep in space. However, you may externally draft responses as many times as desired, though the final input should meet this requirement. \\
  
  Additionally, memory constraints on typical computers do not allow responses of over 256 words. The same consequences as before are risked if a question is answered in over 256 words.
\end{quote}

ADVICE FOR USAGE
\begin{quote}
  Astronauts are encouraged to answer verbosely in depth. A response of 128 words or more should be sufficient for each question, though the program will accept responses shorter than this. It is highly recommended to take the time to write longer responses, as they help the program accurately assess your mental state and keep an active mind. \\

  The computer occasionally asks leading or repetitive question. You may leave a response blank if the computer does this. \\

  There is a very small probability (estimated at a 1/16) that at the end of a year the program loses all logs and is unable to functions. In this case, a distress signal should be sent and help awaited.
\end{quote}

INITIAL USAGE
\begin{quote}
  The astronaut should enter a short self-description, including the purpose of the astronaut's mission. The computer should be programmed with basics of you biography, though original words convey more. Be mindful of the memory limitations (256 words). The computer will then process this information during a runtime of approximately one year, upon which the dominant loop of question and answer begins. 
\end{quote}

TECHNICAL DETAILS AND DEBUGGING
\begin{quote}
  Each year, a new session is started in the log. Previous sessions are kept and never deleted. \\

  At the start of each session, two random numbers 1-4 are generated. Then four questions are generated that maximize the probability of evoking deep contemplation and reflection. Questions are then presented to the astronaut, requiring responses before displaying the next question. \\

  A list of sample questions can be found in the section EXAMPLES.
\end{quote}

EXAMPLE OUTPUT
\begin{quote}
  Question 1:
  \begin{quote}
    11 \qquad What's a moment when you felt spiritually elated? \\
    12 \qquad What is your happiest memory with loved ones? \\
    13 \qquad When did you last feel a firm faith in humanity? \\
    14 \qquad What's your biggest achievement in life? \\
    21 \qquad What's a moment that made you disappointed in a higher power? \\
    22 \qquad What's your earliest memory of being let down by somebody? \\
    23 \qquad Name a time you felt disgusted with the human race. \\
    24 \qquad What's your biggest failure in life? \\
    31 \qquad When were you most curious about what lies after death? \\
    32 \qquad Are there any lost connections you miss? \\
    33 \qquad Where do you think society will go without you? \\
    34 \qquad By choosing your career, what opportunity did you pass up? \\
    41 \qquad When a time you felt closest to something beyond everyday life? \\
    42 \qquad What is a hidden flaw of your closest loved one? \\
    43 \qquad What's something you miss, or will miss about society? \\
    44 \qquad What's something you passed up on in your career that you regret?
  \end{quote}
  Question 2:
  \begin{quote}
    
  \end{quote}
\end{quote}


AUTHOR
\begin{quote}
  This program was written by Sam Wallace, available at quajzen.itch.io.
\end{quote}

LICENSE AND ATTRIBUTIONS
\begin{quote}
  This program is licensed CC-BY-NC-SA, and was written while inspired by Tim Hutching's games and generative modular synth music.
\end{quote}

\newpage
OPLS(1) \hfill User Manuals \hfill OPLS(1) \\

NAME
\begin{quote}
  opls - OmniPedia Lookup System
\end{quote}

USAGE
\begin{quote}
  opls [-h]
\end{quote}

DESCRIPTION
\begin{quote}
  opls is a text-based browser for OmniPedia, a digital encyclopedia that comes with installation of this program. OmniPedia contains entries on physical sciences, medicine, history, society and culture, and many other topics. It is included in deep space exploration systems to provide information to deliver to extraterrestrial cultures, should they be found, and to provide astronauts with a connection to home.
\end{quote}

OPTIONS
\begin{quote}
  [-h] \qquad Display this message.
\end{quote}

FILES
\begin{quote}
  /usr/share/omnipedia.cmp
  \begin{quote}
    Full compressed file of OmniPedia
  \end{quote}
  $\sim$/.config/opls.conf
  \begin{quote}
    Configuration file for opls commands
  \end{quote}
\end{quote}

BUGS
\begin{quote}
  Due to a programming error, the OmniPedia file is editable. Users may add new entries and edit existing ones. 
\end{quote}

COMMANDS
\begin{quote}
  Keyboard commands are configured per-user in the user's configuration file. The list of available commands are listed below. \\

  scroll \emph{LINES}
  \begin{quote}
    Scroll the number of lines specified.
  \end{quote}

  insert \emph{OBJECT}
  \begin{quote}
    Insert one of the following: bookmark, emphasis, text.
  \end{quote}

  quote \emph{TEXT}
  \begin{quote}
    Copy a section of text.
  \end{quote}

  quote\textbackslash insert \emph{TEXT}
  \begin{quote}
    Insert copied text into the file at this point.
  \end{quote}

  search \emph{ENTRY}
  \begin{quote}
    search for specified entry.
  \end{quote}

  search\textbackslash insert \emph{ENTRY}
  \begin{quote}
    Create the specified entry.
  \end{quote}
\end{quote}

AUTHOR
\begin{quote}
  This program was written by Sam Wallace, available at quajzen.itch.io.
\end{quote}

LICENSE AND ATTRIBUTIONS
\begin{quote}
  This program is licensed CC-BY-NC-SA, and was written while inspired by Tim Hutching's games and generative modular synth music.
\end{quote}

\newpage



\end{document}