\documentclass{article}

\usepackage{accanthis}

\usepackage{microtype}
\usepackage[margin=1.25in]{geometry}
\usepackage{multicol}
\usepackage{hyperref}
\setlength{\columnsep}{0.75in}

\pagestyle{empty}

\linespread{1.2}

\title{Goonsspiel}

\begin{document}

{\centering \Large \textbf{Goonsspiel}} \hfill quajzen \hrule

\begin{multicols}{2}

  \noindent  {\large \textbf{Playing the Game}}
  
  One player acts as referee, describing the world, its inhabitants and dangers, and all other players control characters in this world. \\

\noindent  {\large \textbf{Character Creation}}

  Decide on 3 specific skills that your character is adept at.
  Have players come up with these skills themselves.
  Come up with a personal goal for your character, and a connection to an NPC.
  Come up with 3 rather mundane equipment that your character starts with.
  You have a base inventory size of 4, which may be increased by other equipment (such as a backpack). \\

\noindent  {\large \textbf{Risky Encounters}} 

  If there is reasonable doubt that an action might succeed and failure would be a setback, roll 2d6.
  Add one to the roll for each item, skill, or factor that contributes to the success.
  Total up the number of factors countering success and add that number to 10 to create the target.
  If your modified roll is greater than or equal to the target, you succeed.
  Otherwise, you are unsuccessful. \\

  \noindent {\large \textbf{Violence}}

  Every character can take 6 hits; add to this number for armor pieces equipped or relevant defensive skills.
  Usual attacks can deal 1-4 hits; agree on a number that makes sense based on character skills and equipment.
  Players may attempt non--damage maneuvers by sacrificing any hits that would be dealt. \\
  
  \noindent {\large \textbf{Referee Preparation}} 


  Encourage players to think in ways that make the game world interactive through your design of the world, its features and peoples, and adventures.
  Use any supplementary media, tools, tables, or procedures to ease your preparation, but never sacrifice interesting content.
  Prepare open--ended situations that players can resolve in many ways.
  Create common--sense obstacles that can be solved similarly.
  Respect their decisions to manipulate your world in a way that improves the gameplay. \\

  \noindent {\large \textbf{Advancement}} 
  
  Offer opportunities for new skills, interesting items, or engaging powers to player characters.
  Collaborate with them to come up with rewards and advancement that fits the type of game that has been created.
  Prefer non--mechanical advancement. \\

  \noindent {\large \textbf{Freeform Gameplay}} 

  Collaborate on game tone, world elements, pacing, abstraction of actions, granular description, and overall fiction to create a game that everyone enjoys.
  Have players throw in details or suggest outcomes that are non--obvious.
  Decide as a group how much narrative authority the referee should have.
  Speed up, slow down, and manipulate descriptions as desired. \\

  \noindent {\large \textbf{Credits}} 

  Love of the following games have inspired me: Tunnel Goons, 24xx, Primeval 2d6, Skorne, and other systems and discussion among the FKR community.

  


\end{multicols} \hrule
 \vfill
\end{document}